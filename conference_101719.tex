\documentclass[conference]{IEEEtran}
% \IEEEoverridecommandlockouts
% The preceding line is only needed to identify funding in the first footnote. If that is unneeded, please comment it out.
\usepackage{cite}
\usepackage{amsmath,amssymb,amsfonts}
\usepackage{algorithmic}
\usepackage{graphicx}
\usepackage{textcomp}
\usepackage{xcolor}
\def\BibTeX{{\rm B\kern-.05em{\sc i\kern-.025em b}\kern-.08em
    T\kern-.1667em\lower.7ex\hbox{E}\kern-.125emX}}
\begin{document}

\title{LRU*\\
{\footnotesize \textsuperscript{*}}
\thanks{}
}

\author{\IEEEauthorblockN{1\textsuperscript{st} Penghui Zhu}
\IEEEauthorblockA{\textit{dept. name of organization (of Aff.)} \\
\textit{name of organization (of Aff.)}\\
City, Country \\
email address or ORCID}
}

\maketitle

\begin{abstract}
In the contemporary computer system world, Least Recently Used (LRU) algorithm has been a famous and widely adopted cache replacement policy that offers good object hit ratio (OHR) and performance. It maintains the recency information about the objects stored and do replacement based on that knowledge. However, the algorithm is totally based on recency and has not taken the size of the objects into consideration. This can lead to large cache size in memory, causing problems and posing challenges in scenarios where memory is limited. In this paper, a series of modifications to the basic LRU algorithm are introduced, aiming to reduce the overall memory consumption while still reaching a good OHR and performance while keep an eye on the object size when adopting. The Size-Aware CLOCK (SA-CLOCK) algorithm is proved through experiments based on real-world traces, showing that it provides a more memory-efficient cache replacement policy, opening the possibility of maintaining good OHR while reducing memory usage.
\end{abstract}

\begin{IEEEkeywords}
cache, LRU, memory-efficient, size-aware
\end{IEEEkeywords}

\section{Introduction}
Introduction:

In the realm of memory management and data caching, the Least Recently Used (LRU) algorithm has long stood as a cornerstone for optimizing data retrieval in various computing systems. Specifically designed for in-memory cache systems, LRU employs a mechanism that prioritizes recently accessed data, thereby enhancing cache hit rates and consequently improving overall system performance.

However, as computing environments evolve, challenges arise, particularly in scenarios where memory resources are constrained. In such instances, striking a balance between maintaining high cache hit rates and minimizing the memory footprint becomes imperative. Traditional algorithms, including LRU, often fall short in addressing the nuanced intricacies of memory limitations. Some modified algorithms like CLOCK pays more attention to memory usage by simplifying LRU, yet they still neglects object sizes within the cache.

This paper delves into the evolving landscape of caching strategies, motivated by the need to optimize memory usage while preserving cache efficiency. The focus lies on recognizing the constraints posed by limited memory availability and the necessity to rethink existing algorithms to adapt to these constraints.

One notable oversight in conventional caching approaches is the neglect of the individual sizes of objects within the cache. Recognizing this limitation, our proposed algorithm seeks to rectify this gap by introducing a novel approach that not only upholds cache hit rates but also strategically attends to the size of objects stored in the cache.

In the subsequent sections, we present an in-depth exploration of the limitations of traditional algorithms and articulate the design principles and mechanisms behind our newly proposed algorithm. Through meticulous consideration of object sizes, our algorithm aims to demonstrate a reduction in overall memory consumption while keeping cache hit rates at par with or potentially surpassing existing approaches. This paradigm shift represents a significant stride towards more resource-efficient and effective memory management in contemporary computing environments.

The rest of the paper is organized as follows. Section 2 presents the background and motivations. Section 3 elaborates the principle of SA-CLOCK. Section 4 provides evaluation for SA-CLOCK. Section 5 summarizes some related works. The last section concludes this work.

\section{Background and Motivation}

\subsection{Least Recently Used}
The Least Recently Used (LRU) cache replacement policy is a foundational concept in computer science, vital for optimizing cache system performance. Caches, high-speed memory storage units, store frequently accessed data to expedite retrieval from slower, main memory. LRU efficiently manages cache contents by prioritizing the retention of recently accessed data.
LRU operates on the principle of discarding the least recently used items in the cache when space is needed for new data. This strategy is grounded in the concept of temporal locality, assuming that recently accessed items are more likely to be accessed again soon. LRU employs heuristics based on historical usage patterns to make informed decisions about cache management.
Common LRU implementations involve maintaining a linked list or a similar data structure to track the order in which items were accessed. When an item is accessed, it is moved to the front of the list, signifying recent use. Eviction occurs by removing the item at the list's end when the cache reaches capacity. Alternatively, counters or timestamps associated with each item can be used to identify the least recently used item when eviction is necessary.
In implementing the LRU cache policy, a hash map is commonly used to efficiently reference objects in the linked list. This hash map acts as a lookup table, associating each item with its position in the list. However, this approach introduces potential memory overhead as the hash map itself requires extra memory for mapping information. In scenarios with numerous cached items, managing this additional memory becomes crucial to strike a balance between performance gains and resource utilization.
Moreover, LRU also has no adoption or eviction policies related to the objects' size. This means that LRU can adopt very large objects into cache, resulting in the surge of the memory consumption. Given a limited amount of memory space, LRU would hurt other applications for taking up too much memory.

\subsection{CLOCK}
CLOCK is another widely adopted cache replacement policy. CLOCK simplifies memory management by assigning a single-byte flag to each cached object, eliminating the need for complex hashmap structures. The algorithm builds upon LRU principles but distinguishes itself through a circular buffer mechanism, enhancing efficiency without compromising performance.
CLOCK's core principle involves a circular buffer and a clock hand that moves sequentially through cached objects. When an object is accessed, its flag is set, and the clock hand advances. If the hand encounters a flagged object, it resets the flag. This modification, combined with a straightforward replacement strategy based on the clock hand's position, preserves the essence of recency based eviction without the need for extensive tracking. 
CLOCK achieves remarkable efficiency by utilizing only a single byte per cached object for the flag, in stark contrast to LRU's reliance on hashmaps. The circular buffer's minimal extra memory requirement ensures a lightweight footprint. However, CLOCK ignores object characteristics and sizes in its decision-making process. Therefore, CLOCK still suffers from the possibility of taking up too much memory space despite its relatively simpleness compared to LRU. 

\section{Related Works}
In this section, we briefly review recent work related to our design and highlight differences with our approach.

\subsection{TBF}
TBF is new RAM-frugal cache replacement policy that approximates the least-recently-used (LRU) policy. It uses two in-memory Bloom sub-filters (TBF) for maintaining the recency information and leverages an on-flash key–value store to cache objects. TBF requires only one byte of RAM per cached object, making it suitable for implementing very large flash-based caches. The two bloom filters here is used to realize the delete operation as one is discarded periodically. TBF is evaluated through simulation on traces from several block stores and key–value stores, as well as using the Yahoo! Cloud Serving Benchmark in a real system implementation. The results show that TBF achieves cache hit rate and operations per second comparable to those of LRU in spite of its much smaller memory requirements.

\subsection{Me-CLOCK}
Me-CLOCK stands for Memory-Efficient CLOCK and it targets to reduce the memory= overhead introduced by the replacement policy of SSD-based cache. It proposes a memory-efficient framework which keeps most data structures in SSD while just leaving the memory efficient data structure in main memory.
The memory efficient data structure here refers to a new kind of bloom filter introduced by this paper. Unlike a traditional bloom filter, the new bloom filter supports element deletion and it takes over the responsibility of keeping the reused flag, indicating whether a page has been frequently accessed. The framework can be used to implement any LRU-based replacement policies under negligible memory overhead. The evaluation shows that its memory overhead is 10 times less that that introduced by traditional manners such as LRU or CLOCK.
Me-CLOCK is similar to TBF, but it is better in the way that it can be applied to algorithms other than LRU. It also has less memory overhead because it only uses 1 bloom filter instead of 2. 

Both TBF and Me-CLOCK are designed for flash-based cache and have indeed reduced the memory overhead, but they do not pay any attention to the objects themselves. Therefore, they could lead to a bigger memory consumption under the same Object Hit Ratio.

\subsection{AdaptSize}
This paper proposed a new cache admission policy called AdaptSize. It is the first caching system to uses a size-aware admission policy that is continuously adapted to the request traffic. There are two variants proposed. AdaptSize(Chance) probabilistically decides whether or not to admit an object into the cache with the admission probability equal to $e^{-\frac{\text{size}}{c}}$, where c is a tunable parameter. AdaptSize(Threshold) admits an object if its size is below a tunable parameter c. The threshold and its parameters are constantly updated as the requests come in. AdaptSize is evaluated on production request traces and the results show that AdaptSize indeed improves the Object Hit Ratio.
AdaptSize is differ from other policies due to its attention to object size. It chooses to use the more complicated AdaptSize(Chance) while the results shows that there is actually very little different between it and AdaptSize(Threshold).

\section{Prepare Your Paper Before Styling}
Before you begin to format your paper, first write and save the content as a 
separate text file. Complete all content and organizational editing before 
formatting. Please note sections \ref{AA}--\ref{SCM} below for more information on 
proofreading, spelling and grammar.

Keep your text and graphic files separate until after the text has been 
formatted and styled. Do not number text heads---{\LaTeX} will do that 
for you.

\subsection{Abbreviations and Acronyms}\label{AA}
Define abbreviations and acronyms the first time they are used in the text, 
even after they have been defined in the abstract. Abbreviations such as 
IEEE, SI, MKS, CGS, ac, dc, and rms do not have to be defined. Do not use 
abbreviations in the title or heads unless they are unavoidable.

\subsection{Units}
\begin{itemize}
\item Use either SI (MKS) or CGS as primary units. (SI units are encouraged.) English units may be used as secondary units (in parentheses). An exception would be the use of English units as identifiers in trade, such as ``3.5-inch disk drive''.
\item Avoid combining SI and CGS units, such as current in amperes and magnetic field in oersteds. This often leads to confusion because equations do not balance dimensionally. If you must use mixed units, clearly state the units for each quantity that you use in an equation.
\item Do not mix complete spellings and abbreviations of units: ``Wb/m\textsuperscript{2}'' or ``webers per square meter'', not ``webers/m\textsuperscript{2}''. Spell out units when they appear in text: ``. . . a few henries'', not ``. . . a few H''.
\item Use a zero before decimal points: ``0.25'', not ``.25''. Use ``cm\textsuperscript{3}'', not ``cc''.)
\end{itemize}

\subsection{Equations}
Number equations consecutively. To make your 
equations more compact, you may use the solidus (~/~), the exp function, or 
appropriate exponents. Italicize Roman symbols for quantities and variables, 
but not Greek symbols. Use a long dash rather than a hyphen for a minus 
sign. Punctuate equations with commas or periods when they are part of a 
sentence, as in:
\begin{equation}
a+b=\gamma\label{eq}
\end{equation}

Be sure that the 
symbols in your equation have been defined before or immediately following 
the equation. Use ``\eqref{eq}'', not ``Eq.~\eqref{eq}'' or ``equation \eqref{eq}'', except at 
the beginning of a sentence: ``Equation \eqref{eq} is . . .''

\subsection{\LaTeX-Specific Advice}

Please use ``soft'' (e.g., \verb|\eqref{Eq}|) cross references instead
of ``hard'' references (e.g., \verb|(1)|). That will make it possible
to combine sections, add equations, or change the order of figures or
citations without having to go through the file line by line.

Please don't use the \verb|{eqnarray}| equation environment. Use
\verb|{align}| or \verb|{IEEEeqnarray}| instead. The \verb|{eqnarray}|
environment leaves unsightly spaces around relation symbols.

Please note that the \verb|{subequations}| environment in {\LaTeX}
will increment the main equation counter even when there are no
equation numbers displayed. If you forget that, you might write an
article in which the equation numbers skip from (17) to (20), causing
the copy editors to wonder if you've discovered a new method of
counting.

{\BibTeX} does not work by magic. It doesn't get the bibliographic
data from thin air but from .bib files. If you use {\BibTeX} to produce a
bibliography you must send the .bib files. 

{\LaTeX} can't read your mind. If you assign the same label to a
subsubsection and a table, you might find that Table I has been cross
referenced as Table IV-B3. 

{\LaTeX} does not have precognitive abilities. If you put a
\verb|\label| command before the command that updates the counter it's
supposed to be using, the label will pick up the last counter to be
cross referenced instead. In particular, a \verb|\label| command
should not go before the caption of a figure or a table.

Do not use \verb|\nonumber| inside the \verb|{array}| environment. It
will not stop equation numbers inside \verb|{array}| (there won't be
any anyway) and it might stop a wanted equation number in the
surrounding equation.

\subsection{Some Common Mistakes}\label{SCM}
\begin{itemize}
\item The word ``data'' is plural, not singular.
\item The subscript for the permeability of vacuum $\mu_{0}$, and other common scientific constants, is zero with subscript formatting, not a lowercase letter ``o''.
\item In American English, commas, semicolons, periods, question and exclamation marks are located within quotation marks only when a complete thought or name is cited, such as a title or full quotation. When quotation marks are used, instead of a bold or italic typeface, to highlight a word or phrase, punctuation should appear outside of the quotation marks. A parenthetical phrase or statement at the end of a sentence is punctuated outside of the closing parenthesis (like this). (A parenthetical sentence is punctuated within the parentheses.)
\item A graph within a graph is an ``inset'', not an ``insert''. The word alternatively is preferred to the word ``alternately'' (unless you really mean something that alternates).
\item Do not use the word ``essentially'' to mean ``approximately'' or ``effectively''.
\item In your paper title, if the words ``that uses'' can accurately replace the word ``using'', capitalize the ``u''; if not, keep using lower-cased.
\item Be aware of the different meanings of the homophones ``affect'' and ``effect'', ``complement'' and ``compliment'', ``discreet'' and ``discrete'', ``principal'' and ``principle''.
\item Do not confuse ``imply'' and ``infer''.
\item The prefix ``non'' is not a word; it should be joined to the word it modifies, usually without a hyphen.
\item There is no period after the ``et'' in the Latin abbreviation ``et al.''.
\item The abbreviation ``i.e.'' means ``that is'', and the abbreviation ``e.g.'' means ``for example''.
\end{itemize}
An excellent style manual for science writers is \cite{b7}.

\subsection{Authors and Affiliations}
\textbf{The class file is designed for, but not limited to, six authors.} A 
minimum of one author is required for all conference articles. Author names 
should be listed starting from left to right and then moving down to the 
next line. This is the author sequence that will be used in future citations 
and by indexing services. Names should not be listed in columns nor group by 
affiliation. Please keep your affiliations as succinct as possible (for 
example, do not differentiate among departments of the same organization).

\subsection{Identify the Headings}
Headings, or heads, are organizational devices that guide the reader through 
your paper. There are two types: component heads and text heads.

Component heads identify the different components of your paper and are not 
topically subordinate to each other. Examples include Acknowledgments and 
References and, for these, the correct style to use is ``Heading 5''. Use 
``figure caption'' for your Figure captions, and ``table head'' for your 
table title. Run-in heads, such as ``Abstract'', will require you to apply a 
style (in this case, italic) in addition to the style provided by the drop 
down menu to differentiate the head from the text.

Text heads organize the topics on a relational, hierarchical basis. For 
example, the paper title is the primary text head because all subsequent 
material relates and elaborates on this one topic. If there are two or more 
sub-topics, the next level head (uppercase Roman numerals) should be used 
and, conversely, if there are not at least two sub-topics, then no subheads 
should be introduced.

\subsection{Figures and Tables}
\paragraph{Positioning Figures and Tables} Place figures and tables at the top and 
bottom of columns. Avoid placing them in the middle of columns. Large 
figures and tables may span across both columns. Figure captions should be 
below the figures; table heads should appear above the tables. Insert 
figures and tables after they are cited in the text. Use the abbreviation 
``Fig.~\ref{fig}'', even at the beginning of a sentence.

\begin{table}[htbp]
\caption{Table Type Styles}
\begin{center}
\begin{tabular}{|c|c|c|c|}
\hline
\textbf{Table}&\multicolumn{3}{|c|}{\textbf{Table Column Head}} \\
\cline{2-4} 
\textbf{Head} & \textbf{\textit{Table column subhead}}& \textbf{\textit{Subhead}}& \textbf{\textit{Subhead}} \\
\hline
copy& More table copy$^{\mathrm{a}}$& &  \\
\hline
\multicolumn{4}{l}{$^{\mathrm{a}}$Sample of a Table footnote.}
\end{tabular}
\label{tab1}
\end{center}
\end{table}

\begin{figure}[htbp]
\centerline{\includegraphics{fig1.png}}
\caption{Example of a figure caption.}
\label{fig}
\end{figure}

Figure Labels: Use 8 point Times New Roman for Figure labels. Use words 
rather than symbols or abbreviations when writing Figure axis labels to 
avoid confusing the reader. As an example, write the quantity 
``Magnetization'', or ``Magnetization, M'', not just ``M''. If including 
units in the label, present them within parentheses. Do not label axes only 
with units. In the example, write ``Magnetization (A/m)'' or ``Magnetization 
\{A[m(1)]\}'', not just ``A/m''. Do not label axes with a ratio of 
quantities and units. For example, write ``Temperature (K)'', not 
``Temperature/K''.

\section*{Acknowledgment}

The preferred spelling of the word ``acknowledgment'' in America is without 
an ``e'' after the ``g''. Avoid the stilted expression ``one of us (R. B. 
G.) thanks $\ldots$''. Instead, try ``R. B. G. thanks$\ldots$''. Put sponsor 
acknowledgments in the unnumbered footnote on the first page.

\section*{References}

Please number citations consecutively within brackets \cite{b1}. The 
sentence punctuation follows the bracket \cite{b2}. Refer simply to the reference 
number, as in \cite{b3}---do not use ``Ref. \cite{b3}'' or ``reference \cite{b3}'' except at 
the beginning of a sentence: ``Reference \cite{b3} was the first $\ldots$''

Number footnotes separately in superscripts. Place the actual footnote at 
the bottom of the column in which it was cited. Do not put footnotes in the 
abstract or reference list. Use letters for table footnotes.

Unless there are six authors or more give all authors' names; do not use 
``et al.''. Papers that have not been published, even if they have been 
submitted for publication, should be cited as ``unpublished'' \cite{b4}. Papers 
that have been accepted for publication should be cited as ``in press'' \cite{b5}. 
Capitalize only the first word in a paper title, except for proper nouns and 
element symbols.

For papers published in translation journals, please give the English 
citation first, followed by the original foreign-language citation \cite{b6}.

\begin{thebibliography}{00}
\bibitem{b1} G. Eason, B. Noble, and I. N. Sneddon, ``On certain integrals of Lipschitz-Hankel type involving products of Bessel functions,'' Phil. Trans. Roy. Soc. London, vol. A247, pp. 529--551, April 1955.
\bibitem{b2} J. Clerk Maxwell, A Treatise on Electricity and Magnetism, 3rd ed., vol. 2. Oxford: Clarendon, 1892, pp.68--73.
\bibitem{b3} I. S. Jacobs and C. P. Bean, ``Fine particles, thin films and exchange anisotropy,'' in Magnetism, vol. III, G. T. Rado and H. Suhl, Eds. New York: Academic, 1963, pp. 271--350.
\bibitem{b4} K. Elissa, ``Title of paper if known,'' unpublished.
\bibitem{b5} R. Nicole, ``Title of paper with only first word capitalized,'' J. Name Stand. Abbrev., in press.
\bibitem{b6} Y. Yorozu, M. Hirano, K. Oka, and Y. Tagawa, ``Electron spectroscopy studies on magneto-optical media and plastic substrate interface,'' IEEE Transl. J. Magn. Japan, vol. 2, pp. 740--741, August 1987 [Digests 9th Annual Conf. Magnetics Japan, p. 301, 1982].
\bibitem{b7} M. Young, The Technical Writer's Handbook. Mill Valley, CA: University Science, 1989.
\end{thebibliography}
\vspace{12pt}
\color{red}
IEEE conference templates contain guidance text for composing and formatting conference papers. Please ensure that all template text is removed from your conference paper prior to submission to the conference. Failure to remove the template text from your paper may result in your paper not being published.

\end{document}
